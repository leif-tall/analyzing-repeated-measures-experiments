% Options for packages loaded elsewhere
\PassOptionsToPackage{unicode}{hyperref}
\PassOptionsToPackage{hyphens}{url}
\PassOptionsToPackage{dvipsnames,svgnames,x11names}{xcolor}
%
\documentclass[
  letterpaper,
  DIV=11,
  numbers=noendperiod]{scrartcl}

\usepackage{amsmath,amssymb}
\usepackage{iftex}
\ifPDFTeX
  \usepackage[T1]{fontenc}
  \usepackage[utf8]{inputenc}
  \usepackage{textcomp} % provide euro and other symbols
\else % if luatex or xetex
  \usepackage{unicode-math}
  \defaultfontfeatures{Scale=MatchLowercase}
  \defaultfontfeatures[\rmfamily]{Ligatures=TeX,Scale=1}
\fi
\usepackage{lmodern}
\ifPDFTeX\else  
    % xetex/luatex font selection
\fi
% Use upquote if available, for straight quotes in verbatim environments
\IfFileExists{upquote.sty}{\usepackage{upquote}}{}
\IfFileExists{microtype.sty}{% use microtype if available
  \usepackage[]{microtype}
  \UseMicrotypeSet[protrusion]{basicmath} % disable protrusion for tt fonts
}{}
\makeatletter
\@ifundefined{KOMAClassName}{% if non-KOMA class
  \IfFileExists{parskip.sty}{%
    \usepackage{parskip}
  }{% else
    \setlength{\parindent}{0pt}
    \setlength{\parskip}{6pt plus 2pt minus 1pt}}
}{% if KOMA class
  \KOMAoptions{parskip=half}}
\makeatother
\usepackage{xcolor}
\setlength{\emergencystretch}{3em} % prevent overfull lines
\setcounter{secnumdepth}{-\maxdimen} % remove section numbering
% Make \paragraph and \subparagraph free-standing
\ifx\paragraph\undefined\else
  \let\oldparagraph\paragraph
  \renewcommand{\paragraph}[1]{\oldparagraph{#1}\mbox{}}
\fi
\ifx\subparagraph\undefined\else
  \let\oldsubparagraph\subparagraph
  \renewcommand{\subparagraph}[1]{\oldsubparagraph{#1}\mbox{}}
\fi


\providecommand{\tightlist}{%
  \setlength{\itemsep}{0pt}\setlength{\parskip}{0pt}}\usepackage{longtable,booktabs,array}
\usepackage{calc} % for calculating minipage widths
% Correct order of tables after \paragraph or \subparagraph
\usepackage{etoolbox}
\makeatletter
\patchcmd\longtable{\par}{\if@noskipsec\mbox{}\fi\par}{}{}
\makeatother
% Allow footnotes in longtable head/foot
\IfFileExists{footnotehyper.sty}{\usepackage{footnotehyper}}{\usepackage{footnote}}
\makesavenoteenv{longtable}
\usepackage{graphicx}
\makeatletter
\def\maxwidth{\ifdim\Gin@nat@width>\linewidth\linewidth\else\Gin@nat@width\fi}
\def\maxheight{\ifdim\Gin@nat@height>\textheight\textheight\else\Gin@nat@height\fi}
\makeatother
% Scale images if necessary, so that they will not overflow the page
% margins by default, and it is still possible to overwrite the defaults
% using explicit options in \includegraphics[width, height, ...]{}
\setkeys{Gin}{width=\maxwidth,height=\maxheight,keepaspectratio}
% Set default figure placement to htbp
\makeatletter
\def\fps@figure{htbp}
\makeatother
\newlength{\cslhangindent}
\setlength{\cslhangindent}{1.5em}
\newlength{\csllabelwidth}
\setlength{\csllabelwidth}{3em}
\newlength{\cslentryspacingunit} % times entry-spacing
\setlength{\cslentryspacingunit}{\parskip}
\newenvironment{CSLReferences}[2] % #1 hanging-ident, #2 entry spacing
 {% don't indent paragraphs
  \setlength{\parindent}{0pt}
  % turn on hanging indent if param 1 is 1
  \ifodd #1
  \let\oldpar\par
  \def\par{\hangindent=\cslhangindent\oldpar}
  \fi
  % set entry spacing
  \setlength{\parskip}{#2\cslentryspacingunit}
 }%
 {}
\usepackage{calc}
\newcommand{\CSLBlock}[1]{#1\hfill\break}
\newcommand{\CSLLeftMargin}[1]{\parbox[t]{\csllabelwidth}{#1}}
\newcommand{\CSLRightInline}[1]{\parbox[t]{\linewidth - \csllabelwidth}{#1}\break}
\newcommand{\CSLIndent}[1]{\hspace{\cslhangindent}#1}

\usepackage{booktabs}
\usepackage{caption}
\usepackage{longtable}
\usepackage{colortbl}
\usepackage{array}
\KOMAoption{captions}{tableheading}
\makeatletter
\makeatother
\makeatletter
\makeatother
\makeatletter
\@ifpackageloaded{caption}{}{\usepackage{caption}}
\AtBeginDocument{%
\ifdefined\contentsname
  \renewcommand*\contentsname{Table of contents}
\else
  \newcommand\contentsname{Table of contents}
\fi
\ifdefined\listfigurename
  \renewcommand*\listfigurename{List of Figures}
\else
  \newcommand\listfigurename{List of Figures}
\fi
\ifdefined\listtablename
  \renewcommand*\listtablename{List of Tables}
\else
  \newcommand\listtablename{List of Tables}
\fi
\ifdefined\figurename
  \renewcommand*\figurename{Figure}
\else
  \newcommand\figurename{Figure}
\fi
\ifdefined\tablename
  \renewcommand*\tablename{Table}
\else
  \newcommand\tablename{Table}
\fi
}
\@ifpackageloaded{float}{}{\usepackage{float}}
\floatstyle{ruled}
\@ifundefined{c@chapter}{\newfloat{codelisting}{h}{lop}}{\newfloat{codelisting}{h}{lop}[chapter]}
\floatname{codelisting}{Listing}
\newcommand*\listoflistings{\listof{codelisting}{List of Listings}}
\makeatother
\makeatletter
\@ifpackageloaded{caption}{}{\usepackage{caption}}
\@ifpackageloaded{subcaption}{}{\usepackage{subcaption}}
\makeatother
\makeatletter
\@ifpackageloaded{tcolorbox}{}{\usepackage[skins,breakable]{tcolorbox}}
\makeatother
\makeatletter
\@ifundefined{shadecolor}{\definecolor{shadecolor}{rgb}{.97, .97, .97}}
\makeatother
\makeatletter
\makeatother
\makeatletter
\makeatother
\ifLuaTeX
  \usepackage{selnolig}  % disable illegal ligatures
\fi
\IfFileExists{bookmark.sty}{\usepackage{bookmark}}{\usepackage{hyperref}}
\IfFileExists{xurl.sty}{\usepackage{xurl}}{} % add URL line breaks if available
\urlstyle{same} % disable monospaced font for URLs
\hypersetup{
  pdftitle={Analysere eksperimenter med repeterte forsøk},
  colorlinks=true,
  linkcolor={blue},
  filecolor={Maroon},
  citecolor={Blue},
  urlcolor={Blue},
  pdfcreator={LaTeX via pandoc}}

\title{Analysere eksperimenter med repeterte forsøk}
\author{}
\date{}

\begin{document}
\maketitle
\ifdefined\Shaded\renewenvironment{Shaded}{\begin{tcolorbox}[borderline west={3pt}{0pt}{shadecolor}, breakable, sharp corners, interior hidden, boxrule=0pt, enhanced, frame hidden]}{\end{tcolorbox}}\fi

\hypertarget{introduksjon}{%
\subsection{Introduksjon}\label{introduksjon}}

Styrketreningsprogammer består av mange variabler som i teorien kan
påvirke adaptasjoner. Blant dem er volum, intensitet, frekvens,
pauselengder mellom sett, ernæring, kontraksjonstype og
kontraksjonshastighet. Når vi har så mange variabler vi kan manipulere
er det uendelig mange måter vi kan kombinere dette for å få ulike
treningsprogrammer. Når det gjelder volum er debatten rundt ett sett
kontra flere sett noe som har fått oppmerksomhet (Carpinelli and Otto
1998).

Noen studier viser at større volum er gunstig for både muskelstyrke og
hypertrofi (Sooneste et al. 2013; Radaelli et al. 2015). Likevel er det
også noen som finner at lite volum gir økninger i styrke og masse som er
tilsvarende det som oppnås ved moderat volum (Cannon and Marino 2010;
Mitchell et al. 2012). Spredningen i hva studiene viser er sannsynligvis
på grunn av en kombinasjon av små utvalgsstørrelser og individuelle
forskjeller. Studiedesign der man sammenligner ulikt treningsvolum hos
samme person kan i teorien hjelpe med å håndtere disse begrensningene. I
flere av studiene som ser på ett sett versus tre sett er det også
forskjell i intensitet og hvilke øvelser som er brukt (Marx et al. 2001;
Messier and Dill 1985).

Formålet med analysene i denne rapporten var å sammenligne effekt av ett
og flere sett på både muskelstyrke og hypertrofi. På bakgrunn i de
metodiske utfordringene ved studier som sammenligner ett sett med flere
sett så hypotiseres følgende: Tre sett vil være mer effektivt for å
forbedre maksimal muskelstyrke og økning i muskelmasse sammenlignet med
ett sett.

\hypertarget{metode}{%
\subsection{Metode}\label{metode}}

\hypertarget{forsuxf8kspersoner-og-studiedesign}{%
\subsubsection{Forsøkspersoner og
studiedesign}\label{forsuxf8kspersoner-og-studiedesign}}

Førtien mannlige og kvinnelige deltakere ble rekruttert etter kriteriet
om at de ikke røykte og var mellom 18-40 år. Kriterier for ikke å bli
inkludert var mer enn én ukentlig styrkeøkt siste 12 måneder før
intervensjon, intoleranse for bedøvelse, redusert muskelstyrke pga.
skade og inntak av reseptbelagt medisin som kan påvirke
treningsadaptasjoner. Syv deltakere er ekskludert fra analysene fordi de
ikke fullførte minimum 85 \% av oppsatt trening. Blant deltakerne som er
inkludert rapporterte alle at de hadde erfaring med idrettsaktiviteter.
Tjue deltakere drev med fysisk trening når de meldte seg til studien, 10
av disse drev med sporadisk styrketrening, men ingen mer enn én gang i
uka.

Intervensjonen innebar 12 uker med fullkropp styrketrening, alle
fullførte intervensjonen i løpet av September-November. Benøvelsene ble
utført unilateralt for å tillate differensiering av treningsvolum hos
samme deltaker. For hver deltaker ble beina randomisert til å utføre
øvelser med enten ett eller tre sett, altså gjorde hver deltaker begge
protokollene. Muskelstyrke ble samlet inn ved baseline, underveis (uke
3, 5 og 9) og etter intervensjonen, mens målinger av kroppssammensetning
var før og etter intervensjonen.

\hypertarget{treningsprotokoll}{%
\subsubsection{Treningsprotokoll}\label{treningsprotokoll}}

Benøvelser ble gjennomført i følgende rekkefølge: unilateral benpress,
bencurl og kneekstensjon, som ett sett på ene beinet og tre sett på
andre beinet. Benet som skulle trenes i ett sett ble trent mellom andre
og tredje sett på det andre benet som trente tre sett. Etter
beinøvelsene trente de også to sett av bilateral benkpress, nedtrekk, og
enten skulderpress eller sittende roing (skulderpress og sittende roing
varierte med annenhver økt). Pauselengde mellom settene var 1.5-3
minutter. Treningsmotstanden økte gradvis utover intervensjonen, med
10RM første 2 uker, etterfølgt av 8 RM i 3 uker og 7RM i 7 uker. Etter
den niende økten, ble motstanden redusert på én av de tre øktene som var
hver uke. Reduseringen tilsvarte 90 \% i motstand av forrige økt på den
gitte øvelsen, men med mål om samme antall repetisjoner. Det var minimum
48 timer før neste økt etter styrkeøktene som var med maksimal innsats.
Etter styrkeøktene med redusert motstand var det minst 24 timer til
neste økt. For å sikre umiddelbar restitusjon fikk de en standardisert
drikke etter hver økt med 0.15 g/kg protein, 1.2 g/kg karbohydrater og
0.5 g/kg fett.

\hypertarget{muxe5linger-av-muskelstyrke-og-hypertrofi}{%
\subsubsection{Målinger av muskelstyrke og
hypertrofi}\label{muxe5linger-av-muskelstyrke-og-hypertrofi}}

Maksimal styrke er bestemt som den motstanden man maksimalt klarer en
repetisjon av (1RM) i benpress og kneekstensjon. Før selve testen hadde
de en spesifikk oppvarming med 10, 6 og 3 repetisjoner på henholdsvis
50, 75 og 85 \% av forventet 1RM. Deretter ble 1RM bestemt ved å øke
motstanden progressivt helt til deltakeren ikke lenger klarte å løfte
gjennom hele bevegelsesbanen. Den høyeste motstanden hvor repetisjonen
ble godkjent, er definert som 1RM. De fikk fire til seks forsøk hver.

Ved baseline gjennomførte de testene 2 ganger, med 4 dager mellom. Den
høyeste verdien de oppnådde på disse 2 dagene er brukt i analysene.
Styrketestene var minst 48 timer etter en gjennomført økt ved etter
intervensjonen. Ikke alle deltakerne (n = 18) gjorde styrketestene
underveis i intervensjonen (uke 2, 5 og 9). Treningen ble prioritert for
resterende deltakere hvis de gikk glipp av testing eller trening pga.
sykdom eller logistiske utfordringer. Derfor er ikke testene underveis
inkludert i analysene for at det skulle være et større utvalg i
analysene. Resultatene før og etter intervensjonen er det som er
analysert.

Kroppssammensetning for bestemmelse av mager muskelmasse er bestemt ved
dual-energy X-ray absorptiometry (DXA) før og etter intervensjonen
(Lunar Prodigy, GE Healthcare, Oslo, Norway). Før DXA-målinger fikk
deltakerne beskjed om å faste 2 timer og avstå fra krevende fysisk
aktivitet i 48 timer. Det samme med minimum 48 timer fra siste styrkeøkt
til test gjaldt for måling av kroppssammensetning.

\hypertarget{dataanalyser-og-statistikk}{%
\subsubsection{Dataanalyser og
statistikk}\label{dataanalyser-og-statistikk}}

All data er presentert som gjennomsnitt ± standardavvik hvis ikke annet
er oppgitt. Statiske analyser er gjort i R studio (Posit team 2023). Det
er gjort enkle lineære modeller på differansen mellom gruppene (ett sett
\& flere sett) på endringen i styrke og muskelmasse fra før til etter
intervensjonen. For maksimal styrke er det sett på øvelsene benpress og
kneekstensjon. Muskelmasse er målt som endringen i mager muskelmasse i
beinet som har trent ett mot beinet som har trent tre sett.

\hypertarget{resultater}{%
\subsection{Resultater}\label{resultater}}

DXA-resultatene viste at gjennomsnittlig differanse mellom ett og tre
sett var 122.79 (95 \% KI: {[}8.59-237{]}, p = 0.04). Også for
styrkeøvelsene var forbedringen i 1RM i gjennomsnitt større for det
beinet som hadde trent flere sett. I beinpress var forskjellen 7.22 (95
\% KI: {[}0.9-13.5{]}, p = 0.026), mens for kneekstensjon var det 3.6
(95 \% KI: {[}1.4-5.8{]}, p = 0.002) differanse.

Table~\ref{tbl-utgangsnivå}

\begin{table}

\caption{\label{tbl-utgangsnivå}\textbf{?(caption)}}\begin{minipage}[t]{\linewidth}

{\centering 

\begin{verbatim}
# A tibble: 2 x 3
  sets     m_lean sd_lean
  <fct>     <dbl>   <dbl>
1 multiple  8604.   2033.
2 single    8589    2021.
\end{verbatim}

}

\end{minipage}%
\newline
\begin{minipage}[t]{\linewidth}

{\centering 

\begin{verbatim}
# A tibble: 2 x 3
  sets     m_press sd_press
  <chr>      <dbl>    <dbl>
1 single      218.     76.1
2 multiple    208.     76.4
\end{verbatim}

}

\end{minipage}%
\newline
\begin{minipage}[t]{\linewidth}

{\centering 

\begin{verbatim}
# A tibble: 2 x 3
  sets     m_ext sd_ext
  <chr>    <dbl>  <dbl>
1 single    74.3   25.5
2 multiple  69.2   23.3
\end{verbatim}

}

\end{minipage}%
\newline
\begin{minipage}[t]{\linewidth}
\subcaption{\label{tbl-utgangsnivå-1}}

{\centering 

\setlength{\LTpost}{0mm}
\begin{longtable*}{lrrr}
\toprule
Volum & Muskelmasse (g) & Benpress (kg) & Kneekstensjon (kg) \\ 
\midrule\addlinespace[2.5pt]
multiple & $8,603.5$ &plusmn; $2,032.9$ & $208.1$ &plusmn; $76.4$ & $69.2$ &plusmn; $23.3$ \\ 
single & $8,589.0$ &plusmn; $2,021.0$ & $217.9$ &plusmn; $76.1$ & $74.3$ &plusmn; $25.5$ \\ 
\bottomrule
\end{longtable*}
\begin{minipage}{\linewidth}
\emph{Data er presentert som gjennomsnitt ± standardavvik.}\\
\end{minipage}

}

\end{minipage}%

\end{table}

\hypertarget{diskusjon}{%
\subsection{Diskusjon}\label{diskusjon}}

feilkilde dxa: nøyktig hvor på kroppen bildet ble tatt har betydning for
resultatet

\hypertarget{refs}{}
\begin{CSLReferences}{1}{0}
\leavevmode\vadjust pre{\hypertarget{ref-cannon2010}{}}%
Cannon, Jack, and Frank E. Marino. 2010. {``Early-Phase Neuromuscular
Adaptations to High- and Low-Volume Resistance Training in Untrained
Young and Older Women.''} \emph{Journal of Sports Sciences} 28 (14):
1505--14. \url{https://doi.org/10.1080/02640414.2010.517544}.

\leavevmode\vadjust pre{\hypertarget{ref-carpinelli1998}{}}%
Carpinelli, Ralph N., and Robert M. Otto. 1998. {``Strength Training:
Single Versus Multiple Sets.''} \emph{Sports Medicine} 26 (2): 73--84.
\url{https://doi.org/10.2165/00007256-199826020-00002}.

\leavevmode\vadjust pre{\hypertarget{ref-marx2001}{}}%
Marx, James O., Nicholas A. Ratamess, Bradley C. Nindl, Lincoln A.
Gotshalk, Jeff S. Volek, Keiichiro Dohi, Jill A. Bush, et al. 2001.
{``Low-Volume Circuit Versus High-Volume Periodized Resistance Training
in Women:''} \emph{Medicine and Science in Sports and Exercise}, April,
635--43. \url{https://doi.org/10.1097/00005768-200104000-00019}.

\leavevmode\vadjust pre{\hypertarget{ref-messier1985}{}}%
Messier, Stephen P., and Mary Elizabeth Dill. 1985. {``Alterations in
Strength and Maximal Oxygen Uptake Consequent to Nautilus Circuit Weight
Training.''} \emph{Research Quarterly for Exercise and Sport} 56 (4):
345--51. \url{https://doi.org/10.1080/02701367.1985.10605339}.

\leavevmode\vadjust pre{\hypertarget{ref-mitchell2012}{}}%
Mitchell, Cameron J., Tyler A. Churchward-Venne, Daniel W. D. West,
Nicholas A. Burd, Leigh Breen, Steven K. Baker, and Stuart M. Phillips.
2012. {``Resistance Exercise Load Does Not Determine Training-Mediated
Hypertrophic Gains in Young Men.''} \emph{Journal of Applied Physiology}
113 (1): 71--77. \url{https://doi.org/10.1152/japplphysiol.00307.2012}.

\leavevmode\vadjust pre{\hypertarget{ref-rstudio}{}}%
Posit team. 2023. \emph{RStudio: Integrated Development Environment for
r}. Boston, MA: Posit Software, PBC. \url{http://www.posit.co/}.

\leavevmode\vadjust pre{\hypertarget{ref-radaelli2015}{}}%
Radaelli, Regis, Steven J. Fleck, Thalita Leite, Richard D. Leite, Ronei
S. Pinto, Liliam Fernandes, and Roberto Simão. 2015. {``Dose-Response of
1, 3, and 5 Sets of Resistance Exercise on Strength, Local Muscular
Endurance, and Hypertrophy.''} \emph{Journal of Strength and
Conditioning Research} 29 (5): 1349--58.
\url{https://doi.org/10.1519/JSC.0000000000000758}.

\leavevmode\vadjust pre{\hypertarget{ref-sooneste2013}{}}%
Sooneste, Heiki, Michiya Tanimoto, Ryo Kakigi, Norio Saga, and Shizuo
Katamoto. 2013. {``Effects of Training Volume on Strength and
Hypertrophy in Young Men.''} \emph{Journal of Strength and Conditioning
Research} 27 (1): 8--13.
\url{https://doi.org/10.1519/JSC.0b013e3182679215}.

\end{CSLReferences}



\end{document}
